\documentclass[11pt]{article}

    
\usepackage[breakable]{tcolorbox}
    \usepackage{parskip} % Stop auto-indenting (to mimic markdown behaviour)
    
    \usepackage{iftex}
    \ifPDFTeX
    	\usepackage[T1]{fontenc}
    	\usepackage{mathpazo}
    \else
    	\usepackage{fontspec}
    \fi

    % Basic figure setup, for now with no caption control since it's done
    % automatically by Pandoc (which extracts ![](path) syntax from Markdown).
    \usepackage{graphicx}
    % Maintain compatibility with old templates. Remove in nbconvert 6.0
    \let\Oldincludegraphics\includegraphics
    % Ensure that by default, figures have no caption (until we provide a
    % proper Figure object with a Caption API and a way to capture that
    % in the conversion process - todo).
    \usepackage{caption}
    \DeclareCaptionFormat{nocaption}{}
    \captionsetup{format=nocaption,aboveskip=0pt,belowskip=0pt}

    \usepackage[Export]{adjustbox} % Used to constrain images to a maximum size
    \adjustboxset{max size={0.9\linewidth}{0.9\paperheight}}
    \usepackage{float}
    \floatplacement{figure}{H} % forces figures to be placed at the correct location
    \usepackage{xcolor} % Allow colors to be defined
    \usepackage{enumerate} % Needed for markdown enumerations to work
    \usepackage{geometry} % Used to adjust the document margins
    \usepackage{amsmath} % Equations
    \usepackage{amssymb} % Equations
    \usepackage{textcomp} % defines textquotesingle
    % Hack from http://tex.stackexchange.com/a/47451/13684:
    \AtBeginDocument{%
        \def\PYZsq{\textquotesingle}% Upright quotes in Pygmentized code
    }
    \usepackage{upquote} % Upright quotes for verbatim code
    \usepackage{eurosym} % defines \euro
    \usepackage[mathletters]{ucs} % Extended unicode (utf-8) support
    \usepackage{fancyvrb} % verbatim replacement that allows latex
    \usepackage{grffile} % extends the file name processing of package graphics 
                         % to support a larger range
    \makeatletter % fix for grffile with XeLaTeX
    \def\Gread@@xetex#1{%
      \IfFileExists{"\Gin@base".bb}%
      {\Gread@eps{\Gin@base.bb}}%
      {\Gread@@xetex@aux#1}%
    }
    \makeatother

    % The hyperref package gives us a pdf with properly built
    % internal navigation ('pdf bookmarks' for the table of contents,
    % internal cross-reference links, web links for URLs, etc.)
    \usepackage{hyperref}
    % The default LaTeX title has an obnoxious amount of whitespace. By default,
    % titling removes some of it. It also provides customization options.
    \usepackage{titling}
    \usepackage{longtable} % longtable support required by pandoc >1.10
    \usepackage{booktabs}  % table support for pandoc > 1.12.2
    \usepackage[inline]{enumitem} % IRkernel/repr support (it uses the enumerate* environment)
    \usepackage[normalem]{ulem} % ulem is needed to support strikethroughs (\sout)
                                % normalem makes italics be italics, not underlines
    \usepackage{mathrsfs}
    

\usepackage{nopageno}



    
    % Colors for the hyperref package
    \definecolor{urlcolor}{rgb}{0,.145,.698}
    \definecolor{linkcolor}{rgb}{.71,0.21,0.01}
    \definecolor{citecolor}{rgb}{.12,.54,.11}

    % ANSI colors
    \definecolor{ansi-black}{HTML}{3E424D}
    \definecolor{ansi-black-intense}{HTML}{282C36}
    \definecolor{ansi-red}{HTML}{E75C58}
    \definecolor{ansi-red-intense}{HTML}{B22B31}
    \definecolor{ansi-green}{HTML}{00A250}
    \definecolor{ansi-green-intense}{HTML}{007427}
    \definecolor{ansi-yellow}{HTML}{DDB62B}
    \definecolor{ansi-yellow-intense}{HTML}{B27D12}
    \definecolor{ansi-blue}{HTML}{208FFB}
    \definecolor{ansi-blue-intense}{HTML}{0065CA}
    \definecolor{ansi-magenta}{HTML}{D160C4}
    \definecolor{ansi-magenta-intense}{HTML}{A03196}
    \definecolor{ansi-cyan}{HTML}{60C6C8}
    \definecolor{ansi-cyan-intense}{HTML}{258F8F}
    \definecolor{ansi-white}{HTML}{C5C1B4}
    \definecolor{ansi-white-intense}{HTML}{A1A6B2}
    \definecolor{ansi-default-inverse-fg}{HTML}{FFFFFF}
    \definecolor{ansi-default-inverse-bg}{HTML}{000000}

    % commands and environments needed by pandoc snippets
    % extracted from the output of `pandoc -s`
    \providecommand{\tightlist}{%
      \setlength{\itemsep}{0pt}\setlength{\parskip}{0pt}}
    \DefineVerbatimEnvironment{Highlighting}{Verbatim}{commandchars=\\\{\}}
    % Add ',fontsize=\small' for more characters per line
    \newenvironment{Shaded}{}{}
    \newcommand{\KeywordTok}[1]{\textcolor[rgb]{0.00,0.44,0.13}{\textbf{{#1}}}}
    \newcommand{\DataTypeTok}[1]{\textcolor[rgb]{0.56,0.13,0.00}{{#1}}}
    \newcommand{\DecValTok}[1]{\textcolor[rgb]{0.25,0.63,0.44}{{#1}}}
    \newcommand{\BaseNTok}[1]{\textcolor[rgb]{0.25,0.63,0.44}{{#1}}}
    \newcommand{\FloatTok}[1]{\textcolor[rgb]{0.25,0.63,0.44}{{#1}}}
    \newcommand{\CharTok}[1]{\textcolor[rgb]{0.25,0.44,0.63}{{#1}}}
    \newcommand{\StringTok}[1]{\textcolor[rgb]{0.25,0.44,0.63}{{#1}}}
    \newcommand{\CommentTok}[1]{\textcolor[rgb]{0.38,0.63,0.69}{\textit{{#1}}}}
    \newcommand{\OtherTok}[1]{\textcolor[rgb]{0.00,0.44,0.13}{{#1}}}
    \newcommand{\AlertTok}[1]{\textcolor[rgb]{1.00,0.00,0.00}{\textbf{{#1}}}}
    \newcommand{\FunctionTok}[1]{\textcolor[rgb]{0.02,0.16,0.49}{{#1}}}
    \newcommand{\RegionMarkerTok}[1]{{#1}}
    \newcommand{\ErrorTok}[1]{\textcolor[rgb]{1.00,0.00,0.00}{\textbf{{#1}}}}
    \newcommand{\NormalTok}[1]{{#1}}
    
    % Additional commands for more recent versions of Pandoc
    \newcommand{\ConstantTok}[1]{\textcolor[rgb]{0.53,0.00,0.00}{{#1}}}
    \newcommand{\SpecialCharTok}[1]{\textcolor[rgb]{0.25,0.44,0.63}{{#1}}}
    \newcommand{\VerbatimStringTok}[1]{\textcolor[rgb]{0.25,0.44,0.63}{{#1}}}
    \newcommand{\SpecialStringTok}[1]{\textcolor[rgb]{0.73,0.40,0.53}{{#1}}}
    \newcommand{\ImportTok}[1]{{#1}}
    \newcommand{\DocumentationTok}[1]{\textcolor[rgb]{0.73,0.13,0.13}{\textit{{#1}}}}
    \newcommand{\AnnotationTok}[1]{\textcolor[rgb]{0.38,0.63,0.69}{\textbf{\textit{{#1}}}}}
    \newcommand{\CommentVarTok}[1]{\textcolor[rgb]{0.38,0.63,0.69}{\textbf{\textit{{#1}}}}}
    \newcommand{\VariableTok}[1]{\textcolor[rgb]{0.10,0.09,0.49}{{#1}}}
    \newcommand{\ControlFlowTok}[1]{\textcolor[rgb]{0.00,0.44,0.13}{\textbf{{#1}}}}
    \newcommand{\OperatorTok}[1]{\textcolor[rgb]{0.40,0.40,0.40}{{#1}}}
    \newcommand{\BuiltInTok}[1]{{#1}}
    \newcommand{\ExtensionTok}[1]{{#1}}
    \newcommand{\PreprocessorTok}[1]{\textcolor[rgb]{0.74,0.48,0.00}{{#1}}}
    \newcommand{\AttributeTok}[1]{\textcolor[rgb]{0.49,0.56,0.16}{{#1}}}
    \newcommand{\InformationTok}[1]{\textcolor[rgb]{0.38,0.63,0.69}{\textbf{\textit{{#1}}}}}
    \newcommand{\WarningTok}[1]{\textcolor[rgb]{0.38,0.63,0.69}{\textbf{\textit{{#1}}}}}
    
    
    % Define a nice break command that doesn't care if a line doesn't already
    % exist.
    \def\br{\hspace*{\fill} \\* }
    % Math Jax compatibility definitions
    \def\gt{>}
    \def\lt{<}
    \let\Oldtex\TeX
    \let\Oldlatex\LaTeX
    \renewcommand{\TeX}{\textrm{\Oldtex}}
    \renewcommand{\LaTeX}{\textrm{\Oldlatex}}
    % Document parameters
    % Document title
    \title{Multivariate Linear Least Squares Minimization}
    
\date{}

    
    
    
    
% Pygments definitions
\makeatletter
\def\PY@reset{\let\PY@it=\relax \let\PY@bf=\relax%
    \let\PY@ul=\relax \let\PY@tc=\relax%
    \let\PY@bc=\relax \let\PY@ff=\relax}
\def\PY@tok#1{\csname PY@tok@#1\endcsname}
\def\PY@toks#1+{\ifx\relax#1\empty\else%
    \PY@tok{#1}\expandafter\PY@toks\fi}
\def\PY@do#1{\PY@bc{\PY@tc{\PY@ul{%
    \PY@it{\PY@bf{\PY@ff{#1}}}}}}}
\def\PY#1#2{\PY@reset\PY@toks#1+\relax+\PY@do{#2}}

\expandafter\def\csname PY@tok@w\endcsname{\def\PY@tc##1{\textcolor[rgb]{0.73,0.73,0.73}{##1}}}
\expandafter\def\csname PY@tok@c\endcsname{\let\PY@it=\textit\def\PY@tc##1{\textcolor[rgb]{0.25,0.50,0.50}{##1}}}
\expandafter\def\csname PY@tok@cp\endcsname{\def\PY@tc##1{\textcolor[rgb]{0.74,0.48,0.00}{##1}}}
\expandafter\def\csname PY@tok@k\endcsname{\let\PY@bf=\textbf\def\PY@tc##1{\textcolor[rgb]{0.00,0.50,0.00}{##1}}}
\expandafter\def\csname PY@tok@kp\endcsname{\def\PY@tc##1{\textcolor[rgb]{0.00,0.50,0.00}{##1}}}
\expandafter\def\csname PY@tok@kt\endcsname{\def\PY@tc##1{\textcolor[rgb]{0.69,0.00,0.25}{##1}}}
\expandafter\def\csname PY@tok@o\endcsname{\def\PY@tc##1{\textcolor[rgb]{0.40,0.40,0.40}{##1}}}
\expandafter\def\csname PY@tok@ow\endcsname{\let\PY@bf=\textbf\def\PY@tc##1{\textcolor[rgb]{0.67,0.13,1.00}{##1}}}
\expandafter\def\csname PY@tok@nb\endcsname{\def\PY@tc##1{\textcolor[rgb]{0.00,0.50,0.00}{##1}}}
\expandafter\def\csname PY@tok@nf\endcsname{\def\PY@tc##1{\textcolor[rgb]{0.00,0.00,1.00}{##1}}}
\expandafter\def\csname PY@tok@nc\endcsname{\let\PY@bf=\textbf\def\PY@tc##1{\textcolor[rgb]{0.00,0.00,1.00}{##1}}}
\expandafter\def\csname PY@tok@nn\endcsname{\let\PY@bf=\textbf\def\PY@tc##1{\textcolor[rgb]{0.00,0.00,1.00}{##1}}}
\expandafter\def\csname PY@tok@ne\endcsname{\let\PY@bf=\textbf\def\PY@tc##1{\textcolor[rgb]{0.82,0.25,0.23}{##1}}}
\expandafter\def\csname PY@tok@nv\endcsname{\def\PY@tc##1{\textcolor[rgb]{0.10,0.09,0.49}{##1}}}
\expandafter\def\csname PY@tok@no\endcsname{\def\PY@tc##1{\textcolor[rgb]{0.53,0.00,0.00}{##1}}}
\expandafter\def\csname PY@tok@nl\endcsname{\def\PY@tc##1{\textcolor[rgb]{0.63,0.63,0.00}{##1}}}
\expandafter\def\csname PY@tok@ni\endcsname{\let\PY@bf=\textbf\def\PY@tc##1{\textcolor[rgb]{0.60,0.60,0.60}{##1}}}
\expandafter\def\csname PY@tok@na\endcsname{\def\PY@tc##1{\textcolor[rgb]{0.49,0.56,0.16}{##1}}}
\expandafter\def\csname PY@tok@nt\endcsname{\let\PY@bf=\textbf\def\PY@tc##1{\textcolor[rgb]{0.00,0.50,0.00}{##1}}}
\expandafter\def\csname PY@tok@nd\endcsname{\def\PY@tc##1{\textcolor[rgb]{0.67,0.13,1.00}{##1}}}
\expandafter\def\csname PY@tok@s\endcsname{\def\PY@tc##1{\textcolor[rgb]{0.73,0.13,0.13}{##1}}}
\expandafter\def\csname PY@tok@sd\endcsname{\let\PY@it=\textit\def\PY@tc##1{\textcolor[rgb]{0.73,0.13,0.13}{##1}}}
\expandafter\def\csname PY@tok@si\endcsname{\let\PY@bf=\textbf\def\PY@tc##1{\textcolor[rgb]{0.73,0.40,0.53}{##1}}}
\expandafter\def\csname PY@tok@se\endcsname{\let\PY@bf=\textbf\def\PY@tc##1{\textcolor[rgb]{0.73,0.40,0.13}{##1}}}
\expandafter\def\csname PY@tok@sr\endcsname{\def\PY@tc##1{\textcolor[rgb]{0.73,0.40,0.53}{##1}}}
\expandafter\def\csname PY@tok@ss\endcsname{\def\PY@tc##1{\textcolor[rgb]{0.10,0.09,0.49}{##1}}}
\expandafter\def\csname PY@tok@sx\endcsname{\def\PY@tc##1{\textcolor[rgb]{0.00,0.50,0.00}{##1}}}
\expandafter\def\csname PY@tok@m\endcsname{\def\PY@tc##1{\textcolor[rgb]{0.40,0.40,0.40}{##1}}}
\expandafter\def\csname PY@tok@gh\endcsname{\let\PY@bf=\textbf\def\PY@tc##1{\textcolor[rgb]{0.00,0.00,0.50}{##1}}}
\expandafter\def\csname PY@tok@gu\endcsname{\let\PY@bf=\textbf\def\PY@tc##1{\textcolor[rgb]{0.50,0.00,0.50}{##1}}}
\expandafter\def\csname PY@tok@gd\endcsname{\def\PY@tc##1{\textcolor[rgb]{0.63,0.00,0.00}{##1}}}
\expandafter\def\csname PY@tok@gi\endcsname{\def\PY@tc##1{\textcolor[rgb]{0.00,0.63,0.00}{##1}}}
\expandafter\def\csname PY@tok@gr\endcsname{\def\PY@tc##1{\textcolor[rgb]{1.00,0.00,0.00}{##1}}}
\expandafter\def\csname PY@tok@ge\endcsname{\let\PY@it=\textit}
\expandafter\def\csname PY@tok@gs\endcsname{\let\PY@bf=\textbf}
\expandafter\def\csname PY@tok@gp\endcsname{\let\PY@bf=\textbf\def\PY@tc##1{\textcolor[rgb]{0.00,0.00,0.50}{##1}}}
\expandafter\def\csname PY@tok@go\endcsname{\def\PY@tc##1{\textcolor[rgb]{0.53,0.53,0.53}{##1}}}
\expandafter\def\csname PY@tok@gt\endcsname{\def\PY@tc##1{\textcolor[rgb]{0.00,0.27,0.87}{##1}}}
\expandafter\def\csname PY@tok@err\endcsname{\def\PY@bc##1{\setlength{\fboxsep}{0pt}\fcolorbox[rgb]{1.00,0.00,0.00}{1,1,1}{\strut ##1}}}
\expandafter\def\csname PY@tok@kc\endcsname{\let\PY@bf=\textbf\def\PY@tc##1{\textcolor[rgb]{0.00,0.50,0.00}{##1}}}
\expandafter\def\csname PY@tok@kd\endcsname{\let\PY@bf=\textbf\def\PY@tc##1{\textcolor[rgb]{0.00,0.50,0.00}{##1}}}
\expandafter\def\csname PY@tok@kn\endcsname{\let\PY@bf=\textbf\def\PY@tc##1{\textcolor[rgb]{0.00,0.50,0.00}{##1}}}
\expandafter\def\csname PY@tok@kr\endcsname{\let\PY@bf=\textbf\def\PY@tc##1{\textcolor[rgb]{0.00,0.50,0.00}{##1}}}
\expandafter\def\csname PY@tok@bp\endcsname{\def\PY@tc##1{\textcolor[rgb]{0.00,0.50,0.00}{##1}}}
\expandafter\def\csname PY@tok@fm\endcsname{\def\PY@tc##1{\textcolor[rgb]{0.00,0.00,1.00}{##1}}}
\expandafter\def\csname PY@tok@vc\endcsname{\def\PY@tc##1{\textcolor[rgb]{0.10,0.09,0.49}{##1}}}
\expandafter\def\csname PY@tok@vg\endcsname{\def\PY@tc##1{\textcolor[rgb]{0.10,0.09,0.49}{##1}}}
\expandafter\def\csname PY@tok@vi\endcsname{\def\PY@tc##1{\textcolor[rgb]{0.10,0.09,0.49}{##1}}}
\expandafter\def\csname PY@tok@vm\endcsname{\def\PY@tc##1{\textcolor[rgb]{0.10,0.09,0.49}{##1}}}
\expandafter\def\csname PY@tok@sa\endcsname{\def\PY@tc##1{\textcolor[rgb]{0.73,0.13,0.13}{##1}}}
\expandafter\def\csname PY@tok@sb\endcsname{\def\PY@tc##1{\textcolor[rgb]{0.73,0.13,0.13}{##1}}}
\expandafter\def\csname PY@tok@sc\endcsname{\def\PY@tc##1{\textcolor[rgb]{0.73,0.13,0.13}{##1}}}
\expandafter\def\csname PY@tok@dl\endcsname{\def\PY@tc##1{\textcolor[rgb]{0.73,0.13,0.13}{##1}}}
\expandafter\def\csname PY@tok@s2\endcsname{\def\PY@tc##1{\textcolor[rgb]{0.73,0.13,0.13}{##1}}}
\expandafter\def\csname PY@tok@sh\endcsname{\def\PY@tc##1{\textcolor[rgb]{0.73,0.13,0.13}{##1}}}
\expandafter\def\csname PY@tok@s1\endcsname{\def\PY@tc##1{\textcolor[rgb]{0.73,0.13,0.13}{##1}}}
\expandafter\def\csname PY@tok@mb\endcsname{\def\PY@tc##1{\textcolor[rgb]{0.40,0.40,0.40}{##1}}}
\expandafter\def\csname PY@tok@mf\endcsname{\def\PY@tc##1{\textcolor[rgb]{0.40,0.40,0.40}{##1}}}
\expandafter\def\csname PY@tok@mh\endcsname{\def\PY@tc##1{\textcolor[rgb]{0.40,0.40,0.40}{##1}}}
\expandafter\def\csname PY@tok@mi\endcsname{\def\PY@tc##1{\textcolor[rgb]{0.40,0.40,0.40}{##1}}}
\expandafter\def\csname PY@tok@il\endcsname{\def\PY@tc##1{\textcolor[rgb]{0.40,0.40,0.40}{##1}}}
\expandafter\def\csname PY@tok@mo\endcsname{\def\PY@tc##1{\textcolor[rgb]{0.40,0.40,0.40}{##1}}}
\expandafter\def\csname PY@tok@ch\endcsname{\let\PY@it=\textit\def\PY@tc##1{\textcolor[rgb]{0.25,0.50,0.50}{##1}}}
\expandafter\def\csname PY@tok@cm\endcsname{\let\PY@it=\textit\def\PY@tc##1{\textcolor[rgb]{0.25,0.50,0.50}{##1}}}
\expandafter\def\csname PY@tok@cpf\endcsname{\let\PY@it=\textit\def\PY@tc##1{\textcolor[rgb]{0.25,0.50,0.50}{##1}}}
\expandafter\def\csname PY@tok@c1\endcsname{\let\PY@it=\textit\def\PY@tc##1{\textcolor[rgb]{0.25,0.50,0.50}{##1}}}
\expandafter\def\csname PY@tok@cs\endcsname{\let\PY@it=\textit\def\PY@tc##1{\textcolor[rgb]{0.25,0.50,0.50}{##1}}}

\def\PYZbs{\char`\\}
\def\PYZus{\char`\_}
\def\PYZob{\char`\{}
\def\PYZcb{\char`\}}
\def\PYZca{\char`\^}
\def\PYZam{\char`\&}
\def\PYZlt{\char`\<}
\def\PYZgt{\char`\>}
\def\PYZsh{\char`\#}
\def\PYZpc{\char`\%}
\def\PYZdl{\char`\$}
\def\PYZhy{\char`\-}
\def\PYZsq{\char`\'}
\def\PYZdq{\char`\"}
\def\PYZti{\char`\~}
% for compatibility with earlier versions
\def\PYZat{@}
\def\PYZlb{[}
\def\PYZrb{]}
\makeatother


    % For linebreaks inside Verbatim environment from package fancyvrb. 
    \makeatletter
        \newbox\Wrappedcontinuationbox 
        \newbox\Wrappedvisiblespacebox 
        \newcommand*\Wrappedvisiblespace {\textcolor{red}{\textvisiblespace}} 
        \newcommand*\Wrappedcontinuationsymbol {\textcolor{red}{\llap{\tiny$\m@th\hookrightarrow$}}} 
        \newcommand*\Wrappedcontinuationindent {3ex } 
        \newcommand*\Wrappedafterbreak {\kern\Wrappedcontinuationindent\copy\Wrappedcontinuationbox} 
        % Take advantage of the already applied Pygments mark-up to insert 
        % potential linebreaks for TeX processing. 
        %        {, <, #, %, $, ' and ": go to next line. 
        %        _, }, ^, &, >, - and ~: stay at end of broken line. 
        % Use of \textquotesingle for straight quote. 
        \newcommand*\Wrappedbreaksatspecials {% 
            \def\PYGZus{\discretionary{\char`\_}{\Wrappedafterbreak}{\char`\_}}% 
            \def\PYGZob{\discretionary{}{\Wrappedafterbreak\char`\{}{\char`\{}}% 
            \def\PYGZcb{\discretionary{\char`\}}{\Wrappedafterbreak}{\char`\}}}% 
            \def\PYGZca{\discretionary{\char`\^}{\Wrappedafterbreak}{\char`\^}}% 
            \def\PYGZam{\discretionary{\char`\&}{\Wrappedafterbreak}{\char`\&}}% 
            \def\PYGZlt{\discretionary{}{\Wrappedafterbreak\char`\<}{\char`\<}}% 
            \def\PYGZgt{\discretionary{\char`\>}{\Wrappedafterbreak}{\char`\>}}% 
            \def\PYGZsh{\discretionary{}{\Wrappedafterbreak\char`\#}{\char`\#}}% 
            \def\PYGZpc{\discretionary{}{\Wrappedafterbreak\char`\%}{\char`\%}}% 
            \def\PYGZdl{\discretionary{}{\Wrappedafterbreak\char`\$}{\char`\$}}% 
            \def\PYGZhy{\discretionary{\char`\-}{\Wrappedafterbreak}{\char`\-}}% 
            \def\PYGZsq{\discretionary{}{\Wrappedafterbreak\textquotesingle}{\textquotesingle}}% 
            \def\PYGZdq{\discretionary{}{\Wrappedafterbreak\char`\"}{\char`\"}}% 
            \def\PYGZti{\discretionary{\char`\~}{\Wrappedafterbreak}{\char`\~}}% 
        } 
        % Some characters . , ; ? ! / are not pygmentized. 
        % This macro makes them "active" and they will insert potential linebreaks 
        \newcommand*\Wrappedbreaksatpunct {% 
            \lccode`\~`\.\lowercase{\def~}{\discretionary{\hbox{\char`\.}}{\Wrappedafterbreak}{\hbox{\char`\.}}}% 
            \lccode`\~`\,\lowercase{\def~}{\discretionary{\hbox{\char`\,}}{\Wrappedafterbreak}{\hbox{\char`\,}}}% 
            \lccode`\~`\;\lowercase{\def~}{\discretionary{\hbox{\char`\;}}{\Wrappedafterbreak}{\hbox{\char`\;}}}% 
            \lccode`\~`\:\lowercase{\def~}{\discretionary{\hbox{\char`\:}}{\Wrappedafterbreak}{\hbox{\char`\:}}}% 
            \lccode`\~`\?\lowercase{\def~}{\discretionary{\hbox{\char`\?}}{\Wrappedafterbreak}{\hbox{\char`\?}}}% 
            \lccode`\~`\!\lowercase{\def~}{\discretionary{\hbox{\char`\!}}{\Wrappedafterbreak}{\hbox{\char`\!}}}% 
            \lccode`\~`\/\lowercase{\def~}{\discretionary{\hbox{\char`\/}}{\Wrappedafterbreak}{\hbox{\char`\/}}}% 
            \catcode`\.\active
            \catcode`\,\active 
            \catcode`\;\active
            \catcode`\:\active
            \catcode`\?\active
            \catcode`\!\active
            \catcode`\/\active 
            \lccode`\~`\~ 	
        }
    \makeatother

    \let\OriginalVerbatim=\Verbatim
    \makeatletter
    \renewcommand{\Verbatim}[1][1]{%
        %\parskip\z@skip
        \sbox\Wrappedcontinuationbox {\Wrappedcontinuationsymbol}%
        \sbox\Wrappedvisiblespacebox {\FV@SetupFont\Wrappedvisiblespace}%
        \def\FancyVerbFormatLine ##1{\hsize\linewidth
            \vtop{\raggedright\hyphenpenalty\z@\exhyphenpenalty\z@
                \doublehyphendemerits\z@\finalhyphendemerits\z@
                \strut ##1\strut}%
        }%
        % If the linebreak is at a space, the latter will be displayed as visible
        % space at end of first line, and a continuation symbol starts next line.
        % Stretch/shrink are however usually zero for typewriter font.
        \def\FV@Space {%
            \nobreak\hskip\z@ plus\fontdimen3\font minus\fontdimen4\font
            \discretionary{\copy\Wrappedvisiblespacebox}{\Wrappedafterbreak}
            {\kern\fontdimen2\font}%
        }%
        
        % Allow breaks at special characters using \PYG... macros.
        \Wrappedbreaksatspecials
        % Breaks at punctuation characters . , ; ? ! and / need catcode=\active 	
        \OriginalVerbatim[#1,codes*=\Wrappedbreaksatpunct]%
    }
    \makeatother

    % Exact colors from NB
    \definecolor{incolor}{HTML}{303F9F}
    \definecolor{outcolor}{HTML}{D84315}
    \definecolor{cellborder}{HTML}{CFCFCF}
    \definecolor{cellbackground}{HTML}{F7F7F7}
    
    % prompt
    \makeatletter
    \newcommand{\boxspacing}{\kern\kvtcb@left@rule\kern\kvtcb@boxsep}
    \makeatother
    \newcommand{\prompt}[4]{
        \ttfamily\llap{{\color{#2}[#3]:\hspace{3pt}#4}}\vspace{-\baselineskip}
    }
    

    
    % Prevent overflowing lines due to hard-to-break entities
    \sloppy 
    % Setup hyperref package
    \hypersetup{
      breaklinks=true,  % so long urls are correctly broken across lines
      colorlinks=true,
      urlcolor=urlcolor,
      linkcolor=linkcolor,
      citecolor=citecolor,
      }
    % Slightly bigger margins than the latex defaults
    
    \geometry{verbose,tmargin=1in,bmargin=1in,lmargin=1in,rmargin=1in}
    
    

\begin{document}
    
    \maketitle
    
    

    


    \hypertarget{multivariate-linear-least-squares-minimization}{%
\section*{Multivariate Linear Least Squares
Minimization}\label{multivariate-linear-least-squares-minimization}}





    In \href{/linear-regression/least-squares}{Linear Least Squares
Minimization}, we considered the linear functional relation between two
measurable variables, \(x\) and \(y\):

\[
y = a_0 + a_1 x
\]

where \(a_0\) and \(a_1\) are unknown conditions to be determined.

On this page we will look at the more generic case, where we solve the
problem for an arbitrary number of variables and constants.





    \hypertarget{three-variables}{%
\subsection*{Three Variables}\label{three-variables}}





    Let's start by solving this problem for three measurable variables:
\(y\), \(x_1\) and \(x_2\), in the linear functional relation:

\[
y = a_0 + a_1 x_1 + a_2 x_2
\]

where \(a_0\), \(a_1\) and \(a_2\) are unknown coefficients.

Consider a data set of measured \((x_{1i}, x_{2i}, y_i)\) pairs for
\(i = 1, 2, 3, \dots , N\). If we attribute the dispersion of this data
from the functional relation to error in the \(y_i\) terms,
\(\epsilon_i\), then we can relate the data points with:

\[
\begin{align*}
y_i + \epsilon_i &= a_0 + a_1 x_{1 i} + a_2 x_{2 i}\\
\therefore \epsilon_i = a_0 + a_1 x_{1 i} + a_2 x_{2 i} - y_i\\
\end{align*}
\]

The sum of errors squared is given by:

\[
\begin{align*}
S &= \sum_{i = 1}^N \epsilon_i^2 \\
  &= \sum_{i=1}^N (a_0 + a_1 x_{1i} + a_2 x_{2i} - y_i)\\
\end{align*}
\]

We want to minimize \(S\) with respect to each of the constants,
\(a_0\), \(a_1\) and \(a_2\):

\[
\frac{\partial S}{\partial a_0} = 2 \sum_{i=0}^n (a_0 + a_1x_{1i} + a_2x_{2i} - y_i) = 0
\] , \[
\frac{\partial S}{\partial a_1} = 2 \sum_{i=0}^n (a_0 + a_1x_{1i} + a_2x_{2i} - y_i)x_{1i} = 0
\] and \[
\frac{\partial S}{\partial a_2} = 2 \sum_{i=0}^n (a_0 + a_1x_{1i} + a_2x_{2i} - y_i)x_{2i} = 0
\]





    Re-arranging the above equations and using our statistical notation
yields:

\[
a_0 + a_1 \langle x_1 \rangle + a_2 \langle x_2 \rangle = \langle y \rangle
\]

,

\[
a_0\langle x_1\rangle + a_1\langle x_1^2\rangle + a_2\langle x_1x_2 \rangle = \langle x_1y \rangle\\
\]

and

\[
a_0\langle x_2\rangle + a_1\langle x_1x_2\rangle + a_2\langle x_2^2\rangle = \langle x_2y\rangle
\]

This time algebraic manipulation is a lot more work, instead we shall
use a matrix equation (which will serve us better in the more generic
case to come). The matrix equation representation is:

\[
\begin{pmatrix}
  1     &\langle x_1\rangle     &\langle{x_2}\rangle\\
  \langle{x_1}\rangle   &\langle{x_1^2}\rangle      &\langle{x_1x_2}\rangle\\
  \langle{x_2}\rangle   &\langle{x_1x_2}\rangle &\langle{x_2^2}\rangle\\
 \end{pmatrix}
  \begin{pmatrix}
  a_0\\
  a_1\\
  a_2\\
 \end{pmatrix}
 = 
 \begin{pmatrix}
  \langle{y}\rangle\\
  \langle{x_1 y}\rangle\\
  \langle{x_2 y}\rangle
 \end{pmatrix}
\]

This can easily be solved numerically using: \[
\begin{align*}
\boldsymbol{X}\boldsymbol{A} &= \boldsymbol{Y}\\ 
\therefore \boldsymbol{A} &= \boldsymbol{X}^{-1} \boldsymbol{Y}\\
\end{align*}
\]





    \hypertarget{example---cepheid-variables}{%
\subsubsection*{Example - Cepheid
Variables}\label{example---cepheid-variables}}





    You now have all you need to find the unknown coefficients for the full
functional relation of the magnitude (\(M\)), period (\(P\)) and color
(\(B-V\)) of the Cepheid variables:

\[
M = a_0 + a_1 \log P + a_2 (B - V)
\]

using the same data file as before. (You should find the values
\(a_0 = -2.15\) mag, \(a_1 = -3.12\) mag and \(a_2 = 1.49\))





    \hypertarget{arbitrarily-many-variables}{%
\subsection*{Arbitrarily Many
Variables}\label{arbitrarily-many-variables}}





    Consider a linear functional relation between measurable variables
\(x_1\), \(x_2\), \(x_3\), \(\dots\), \(x_m\) and \(y\):

\[
\begin{align*}
y   &= a_0 + a_1 x_1 + a_2 x_2 + \dots + a_m x_m\\
    &= a_0 + \sum_{j = 1}^m a_j x_j\\
\end{align*}
\]

where \(a_0\), \(a_1\), \(\dots\) and \(a_m\) are unknown constants.

Suppose we have a data set of measured
\((x_{1i}, x_{2i}, \dots, x_{mi}, y_i)\) values for
\(i = 1, 2, 3, \dots, N\). As before, we assume that the dispersion in
our data from the functional relation is due to error in the \(y_i\)
data points only. Therefore we can write the relation between our data
points as:

\[
y_i + \epsilon_i = a_0 + \sum_{j = 1}^m a_j x_{ji}
\]





    The sum of errors squared can thus be written as:

\[
S = \sum_{i=1}^N \bigg(a_0 + \bigg(\sum_{j=1}^m a_j x_{ji} \bigg) - y_i\bigg)^2
\]





    We want to find the values of \(a_0\), \(a_1\), \(\dots\) and \(a_m\)
which gives us the minimum value of \(S\). Minimizing \(S\) with respect
to \(a_0\) gives us:

\[
\frac{\partial S}{\partial a_0} = 2 \sum_{i=1}^N \bigg(a_0 + \bigg(\sum_{j=1}^m a_j x_{ji} \bigg) - y_i\bigg) = 0
\]

Distributing the sum over \(i\) amongst the terms:

\[
\therefore N a_0 + \bigg(\sum_{j=1}^m a_j \sum_{i=1}^N x_{ji} \bigg) - \sum_{i=1}^N y_i = 0
\]

Dividing by \(N\):

\[
\therefore a_0 + \bigg(\sum_{j=1}^m a_j \frac{1}{N}\sum_{i=1}^N x_{ji} \bigg) - \frac{1}{N}\sum_{i=1}^N y_i = 0
\]

Using our stats notation:

\[
\therefore a_0 + \sum_{j=1}^m a_j \langle{x_{j}}\rangle = \langle{y}\rangle
\]





    Now, let's minimize \(S\) with respect to one of the \(a_k\) for
\(k = 1, 2, \dots, m\), following a similar line of algebraic
manipulation as above:

\[
\begin{align*}
\frac{\partial S}{\partial a_k} = \sum_{i=1}^N 2 x_{ki} \bigg(a_0 + \bigg(\sum_{j=1}^m a_j x_{ji} \bigg) - y_i\bigg) &= 0\\
\therefore a_0 \sum_{i=1}^N x_{ki} + \sum_{j=1}^m a_j \bigg(\sum_{i=1}^N x_{ki} x_{ji} \bigg) - \sum_{i=1}^N x_{ki} y_i &= 0\\
\therefore a_0 \langle{x_{k}}\rangle + \sum_{j=1}^m a_j \langle{x_k x_j}\rangle &= \langle{x_k y}\rangle\\
\end{align*}
\]





    Writing the results for \(a_0\) and \(a_k\) (\(k = 1,\dots,m\)) into a
system of equations, expanding the sum over \(j\):

\[\begin{eqnarray*}
a_0 &+& a_1 \langle{x_1}\rangle &+& a_2 \langle{x_2}\rangle  &+& \dots &+& a_m \langle{x_m}\rangle &=& \langle{y}\rangle\\
a_0 \langle{x_{1}}\rangle &+& a_1 \langle{x_1~^2}\rangle &+& a_2 \langle{x_1 x_2}\rangle &+& \cdots &+& a_m \langle{x_1 x_m}\rangle &=&  \langle{x_1 y}\rangle\\
a_0 \langle{x_{2}}\rangle &+& a_1 \langle{x_2 x_1}\rangle &+& a_2 \langle{x_2~^2}\rangle &+& \cdots &+& a_m \langle{x_2 x_m}\rangle &=&  \langle{x_2 y}\rangle\\
a_0 \langle{x_{3}}\rangle &+& a_1 \langle{x_3 x_1}\rangle &+& a_2 \langle{x_3 x_2}\rangle &+& \cdots &+& a_m \langle{x_3 x_m}\rangle &=&  \langle{x_3 y}\rangle\\
\vdots ~~~~ &+& ~~~~\vdots &+& ~~~~\vdots &+& \ddots &+& ~~~~\vdots &=& ~~~\vdots\\
a_0 \langle{x_{m}}\rangle &+& a_1 \langle{x_m x_1}\rangle &+& a_2 \langle{x_m x_2}\rangle &+& \cdots &+& a_m \langle{x_m~^2}\rangle &=&  \langle{x_m y}\rangle\\
\end{eqnarray*}\]





    To solve these equations numerically, we can reformulate these equations
into a matrix equation:

\[
\begin{pmatrix}
1                     & \langle{x_1}\rangle      & \langle{x_2}\rangle      & \cdots & \langle{x_m}\rangle\\
\langle{x_1}\rangle   & \langle{x_1~^2}\rangle   & \langle{x_1 x_2}\rangle  & \cdots & \langle{x_1 x_m}\rangle\\
\langle{x_2}\rangle   & \langle{x_2 x_1}\rangle  & \langle{x_2~^2}\rangle   & \cdots & \langle{x_2 x_m}\rangle\\
\vdots                & \vdots                   & \vdots                   & \ddots & \vdots\\
\langle{x_m}\rangle   & \langle{x_m x_1}\rangle  & \langle{x_m x_2}\rangle  & \cdots & \langle{x_m~^2}\rangle\\
\end{pmatrix}
\begin{pmatrix}
a_0\\
a_1\\
a_2\\
\vdots\\
a_m\\
\end{pmatrix}
=
\begin{pmatrix}
\langle{y}\rangle\\
\langle{x_1 y}\rangle\\
\langle{x_2 y}\rangle\\
\vdots\\
\langle{x_m y}\rangle\\
\end{pmatrix}
\]





    Notice that the left most matrix is symmetric about the diagonal, this
can come in handy when computing the matrix elements. As before, this
equation can be solved for the \(a_i\) by inverting the left most
matrix, i.e.

\[
\begin{align*}
\boldsymbol{X} \boldsymbol{A} &= \boldsymbol{Y}\\
\therefore \boldsymbol{A} &= \boldsymbol{X}^{-1} \boldsymbol{Y}\\
\end{align*}
\]





    \hypertarget{python-implementation}{%
\subsubsection*{Python Implementation}\label{python-implementation}}





    Let's work on a Python implementation of this solution. You may want to
try it yourself before reading further. In order to verify our
implementation we will use the Cepheid data we've used so far, though in
further exercises you will be given data sets containing more variables.





    We start by reading in the file. We will read the data into a 2D array.
This can be achieved using the standard library as in the
\href{/file-io/data}{\textbf{Data Files}} section in the \textbf{File
I/O} chapter, or using \texttt{numpy.loadtxt()} (documentation
\href{https://numpy.org/doc/stable/reference/generated/numpy.loadtxt.html}{here}).
We shall use the latter as it is far more convenient: 



    \begin{tcolorbox}[breakable, size=fbox, boxrule=1pt, pad at break*=1mm,colback=cellbackground, colframe=cellborder]
\prompt{In}{incolor}{43}{\boxspacing}
\begin{Verbatim}[commandchars=\\\{\}]
\PY{k+kn}{import} \PY{n+nn}{numpy} \PY{k}{as} \PY{n+nn}{np}

\PY{c+c1}{\PYZsh{}\PYZsh{} Reading in the file}

\PY{n}{data} \PY{o}{=} \PY{n}{np}\PY{o}{.}\PY{n}{loadtxt}\PY{p}{(}\PY{l+s+s1}{\PYZsq{}}\PY{l+s+s1}{data/cepheid\PYZus{}data.csv}\PY{l+s+s1}{\PYZsq{}}\PY{p}{,} \PY{n}{delimiter} \PY{o}{=} \PY{l+s+s1}{\PYZsq{}}\PY{l+s+s1}{,}\PY{l+s+s1}{\PYZsq{}}\PY{p}{,} \PY{n}{skiprows} \PY{o}{=} \PY{l+m+mi}{1}\PY{p}{)}
\end{Verbatim}
\end{tcolorbox}



    The \texttt{data} array contains all of the data points for
\(y_i, x_{1i}, x_{2i}, x_{3i}, \dots, x_{ji}, \dots, x_{mi}\), where
\(i = 1, \dots, N\) corresponds to each row of \texttt{data}. Now, we
want the data in the format:

\[
\begin{eqnarray*}
\begin{matrix}
\texttt{data} = [   & [ y_{1},  & x_{11}, & x_{21}, & \cdots, & x_{m1} &], \\
                    & [ y_{2},  & x_{12}, & x_{22}, & \cdots, & x_{m2} &],\\
                    & [ y_{3},  & x_{13}, & x_{23}, & \cdots, & x_{m3} &],\\
                    & [ ~\vdots~, & \vdots~~, & \vdots~~, & \ddots, & \vdots &],\\
                    & [ y_{N},  & x_{1N}, & x_{2N}, & \cdots, & x_{mN} &]]\\
\end{matrix}
\end{eqnarray*}
\]

as this will make slicing it more clear. In the case of the Cepheid
variable data, however, we have our ``\(y\)'' variable in the central
column. Therefore we shall swap column 1 and 0 to better align with our
desired data structure:



    \begin{tcolorbox}[breakable, size=fbox, boxrule=1pt, pad at break*=1mm,colback=cellbackground, colframe=cellborder]
\prompt{In}{incolor}{44}{\boxspacing}
\begin{Verbatim}[commandchars=\\\{\}]
\PY{c+c1}{\PYZsh{} Swapping data[:, 0] and data[:, 1]}
\PY{c+c1}{\PYZsh{} Note that this is particular to the data file we are using}
\PY{c+c1}{\PYZsh{} np.copy is necessaary as arrays are not passed as values by default but as reference}

\PY{n}{data}\PY{p}{[}\PY{p}{:}\PY{p}{,} \PY{l+m+mi}{0}\PY{p}{]}\PY{p}{,} \PY{n}{data}\PY{p}{[}\PY{p}{:}\PY{p}{,} \PY{l+m+mi}{1}\PY{p}{]} \PY{o}{=} \PY{n}{np}\PY{o}{.}\PY{n}{copy}\PY{p}{(}\PY{n}{data}\PY{p}{[}\PY{p}{:}\PY{p}{,} \PY{l+m+mi}{1}\PY{p}{]}\PY{p}{)}\PY{p}{,} \PY{n}{np}\PY{o}{.}\PY{n}{copy}\PY{p}{(}\PY{n}{data}\PY{p}{[}\PY{p}{:}\PY{p}{,} \PY{l+m+mi}{0}\PY{p}{]}\PY{p}{)}
\end{Verbatim}
\end{tcolorbox}



    To extract the values of a single variable for each measurement, slice
columns out of \texttt{data}. For example, the \(y_i\) are contained in
the slice \texttt{data{[}:,\ 0{]}}, the \(x_{1i}\) are contained in
\texttt{data{[}:,\ 1{]}}, the \(x_{2i}\) are contained in
\texttt{data{[}:,\ 2{]}}, etc.





    Note that for each of the sums along the data sets (\(\sum_{i = 1}^N\)),
we will be summing along the columns. For example, for the quantity:

\[
\langle x_1 \rangle = \frac{1}{N} \sum_{i = 1}^N x_{1i}
\]



    \begin{tcolorbox}[breakable, size=fbox, boxrule=1pt, pad at break*=1mm,colback=cellbackground, colframe=cellborder]
\prompt{In}{incolor}{45}{\boxspacing}
\begin{Verbatim}[commandchars=\\\{\}]
\PY{c+c1}{\PYZsh{}Using numpy.mean to calculate the expectation value}
\PY{c+c1}{\PYZsh{}Note that x1 = data[:,1]}

\PY{n}{x1\PYZus{}mean} \PY{o}{=} \PY{n}{np}\PY{o}{.}\PY{n}{mean}\PY{p}{(}\PY{n}{data}\PY{p}{[}\PY{p}{:}\PY{p}{,}\PY{l+m+mi}{1}\PY{p}{]}\PY{p}{)}
\end{Verbatim}
\end{tcolorbox}



    To calculate an expectation value like

\[
\langle x_1 x_2 \rangle = \frac{1}{N} \sum_{i = 1}^N x_{1 i} x_{2 i}
\]

we can use:



    \begin{tcolorbox}[breakable, size=fbox, boxrule=1pt, pad at break*=1mm,colback=cellbackground, colframe=cellborder]
\prompt{In}{incolor}{46}{\boxspacing}
\begin{Verbatim}[commandchars=\\\{\}]
\PY{n}{x1\PYZus{}x2\PYZus{}mean} \PY{o}{=} \PY{n}{np}\PY{o}{.}\PY{n}{mean}\PY{p}{(} \PY{n}{data}\PY{p}{[}\PY{p}{:}\PY{p}{,}\PY{l+m+mi}{1}\PY{p}{]} \PY{o}{*} \PY{n}{data}\PY{p}{[}\PY{p}{:}\PY{p}{,}\PY{l+m+mi}{2}\PY{p}{]} \PY{p}{)}
\end{Verbatim}
\end{tcolorbox}



    where we've made use of NumPy array's vectorized operation to multiply
each element together before taking the mean of the results.





    \hypertarget{constructing-the-boldsymbolx-matrix}{%
\paragraph{\texorpdfstring{Constructing the \(\boldsymbol{X}\)
Matrix}{Constructing the \textbackslash{}boldsymbol\{X\} Matrix}}\label{constructing-the-boldsymbolx-matrix}}





    Before we continue, let's break down the structure of the matrix:

\[
\boldsymbol{X} = 
\begin{pmatrix}
1                     & \langle{x_1}\rangle      & \langle{x_2}\rangle      & \langle{x_3}\rangle      & \cdots & \langle{x_m}\rangle\\
\langle{x_1}\rangle   & \langle{x_1~^2}\rangle   & \langle{x_1 x_2}\rangle  & \langle{x_1 x_3}\rangle  & \cdots & \langle{x_1 x_m}\rangle\\
\langle{x_2}\rangle   & \langle{x_2 x_1}\rangle  & \langle{x_2~^2}\rangle   & \langle{x_2 x_3}\rangle  & \cdots & \langle{x_2 x_m}\rangle\\
\langle{x_3}\rangle   & \langle{x_3 x_1}\rangle  & \langle{x_3 x_2}\rangle  & \langle{x_3~^2}\rangle   & \cdots & \langle{x_3 x_m}\rangle\\
\vdots                & \vdots                   & \vdots                   & \vdots                   & \ddots & \vdots\\
\langle{x_m}\rangle   & \langle{x_m x_1}\rangle  & \langle{x_m x_2}\rangle  & \langle{x_m x_3}\rangle  & \cdots & \langle{x_m~^2}\rangle\\
\end{pmatrix}
\]





    Let's construct an empty matrix for which we will fill in the entries as
we go:



    \begin{tcolorbox}[breakable, size=fbox, boxrule=1pt, pad at break*=1mm,colback=cellbackground, colframe=cellborder]
\prompt{In}{incolor}{47}{\boxspacing}
\begin{Verbatim}[commandchars=\\\{\}]
\PY{n}{var\PYZus{}count} \PY{o}{=} \PY{n}{data}\PY{o}{.}\PY{n}{shape}\PY{p}{[}\PY{l+m+mi}{1}\PY{p}{]}

\PY{n}{X} \PY{o}{=} \PY{n}{np}\PY{o}{.}\PY{n}{matrix}\PY{p}{(}\PY{n}{np}\PY{o}{.}\PY{n}{ones}\PY{p}{(}\PY{p}{(}\PY{n}{var\PYZus{}count}\PY{p}{,} \PY{n}{var\PYZus{}count}\PY{p}{)}\PY{p}{)}\PY{p}{)}
\end{Verbatim}
\end{tcolorbox}



    Note that we have created an \((m+1)\times(m+1)\) matrix, where \(m+1\)
is given by the length of axis-1 of \texttt{data}.





    Now, as we have noted before, \(\boldsymbol{X}\) is a symmetric matrix.
That is for for row \(k\) and column \(l\),
\(\boldsymbol{X}_{k l} = \boldsymbol{X}_{l k}\). We only need to
construct one of the triangles of the matrix, the other is obtained for
free.

Let's work with the upper triangle of the matrix. Here there are 3
regions with distinguishable structures

\begin{enumerate}
\def\labelenumi{\arabic{enumi}.}
\tightlist
\item
  The first row
\item
  The diagonal
\item
  The remaining triangle
\end{enumerate}





    The first element of the matrix is just one. The remainder of the first
row is simply the expectation value of each of the \(x_j\):

\[
\boldsymbol{X}_{0 0} = 1
\]

and

\[
\boldsymbol{X}_{0 l} = \langle{x_l}\rangle ~~~~ \text{where}~ l = 1, 2, \dots, m
\]

Note that here we are indexing \(\boldsymbol{X}\) from 0 to better
translate it to code:



    \begin{tcolorbox}[breakable, size=fbox, boxrule=1pt, pad at break*=1mm,colback=cellbackground, colframe=cellborder]
\prompt{In}{incolor}{48}{\boxspacing}
\begin{Verbatim}[commandchars=\\\{\}]
\PY{c+c1}{\PYZsh{} First row and column}
\PY{c+c1}{\PYZsh{} We leave the first element as is}

\PY{k}{for} \PY{n}{l} \PY{o+ow}{in} \PY{n+nb}{range}\PY{p}{(}\PY{l+m+mi}{1}\PY{p}{,} \PY{n}{var\PYZus{}count}\PY{p}{)}\PY{p}{:}
    \PY{n}{X}\PY{p}{[}\PY{l+m+mi}{0}\PY{p}{,} \PY{n}{l}\PY{p}{]} \PY{o}{=} \PY{n}{np}\PY{o}{.}\PY{n}{mean}\PY{p}{(}\PY{n}{data}\PY{p}{[}\PY{p}{:}\PY{p}{,} \PY{n}{l}\PY{p}{]}\PY{p}{)}
    
    \PY{c+c1}{\PYZsh{} Setting the values for the first column}
    \PY{c+c1}{\PYZsh{} remember that X[k, l] = X[l, k]}
    \PY{n}{X}\PY{p}{[}\PY{n}{l}\PY{p}{,} \PY{l+m+mi}{0}\PY{p}{]} \PY{o}{=} \PY{n}{X}\PY{p}{[}\PY{l+m+mi}{0}\PY{p}{,} \PY{n}{l}\PY{p}{]}
\end{Verbatim}
\end{tcolorbox}



    Now, consider the triangle off of the diagonal. That is the region:





    \[
\begin{pmatrix}
-         & -        & -                         & -                        & \cdots  & - \\
-         & -        & \langle{x_1 x_2}\rangle   & \langle{x_1 x_3}\rangle  & \cdots  & \langle{x_1 x_m}\rangle\\
-         & -        & -                         & \langle{x_2 x_3}\rangle  & \cdots  & \langle{x_2 x_m}\rangle\\
-         & -        & -                         & -                        & \cdots  & \langle{x_3 x_m}\rangle\\
\vdots    & \vdots   & \vdots                    & \vdots                   & \ddots  & \vdots \\
-         & -        & -                         & -                        & \cdots  & \langle{x_m~^2}\rangle\\
\end{pmatrix}
\]





    This region exhibits the pattern:

\[
\boldsymbol{X}_{k l} = \langle{x_k x_l}\rangle ~~~~\text{where}~ l > k
\]

The diagonal has a fairly simple pattern, starting from (row, column)
\((1,1)\):

\[
\boldsymbol{X}_{k k} = \langle{x_k~^2}\rangle
\]

Note, however, that this is a special case of the rules for constructing
region 3. We can therefore combine regions 2 and 3 with the rule:

\[
\boldsymbol{X}_{k l} = \langle{x_k x_l}\rangle ~~~~\text{where}~ l \geq k
\]

In the code this becomes:



    \begin{tcolorbox}[breakable, size=fbox, boxrule=1pt, pad at break*=1mm,colback=cellbackground, colframe=cellborder]
\prompt{In}{incolor}{49}{\boxspacing}
\begin{Verbatim}[commandchars=\\\{\}]
\PY{c+c1}{\PYZsh{} Inner matrix}

\PY{k}{for} \PY{n}{k} \PY{o+ow}{in} \PY{n+nb}{range}\PY{p}{(}\PY{l+m+mi}{1}\PY{p}{,} \PY{n}{var\PYZus{}count}\PY{p}{)}\PY{p}{:}
    \PY{k}{for} \PY{n}{l} \PY{o+ow}{in} \PY{n+nb}{range}\PY{p}{(}\PY{n}{k}\PY{p}{,} \PY{n}{var\PYZus{}count}\PY{p}{)}\PY{p}{:}
        \PY{n}{X}\PY{p}{[}\PY{n}{k}\PY{p}{,} \PY{n}{l}\PY{p}{]} \PY{o}{=} \PY{n}{np}\PY{o}{.}\PY{n}{mean}\PY{p}{(} \PY{n}{data}\PY{p}{[}\PY{p}{:}\PY{p}{,} \PY{n}{k}\PY{p}{]} \PY{o}{*} \PY{n}{data}\PY{p}{[}\PY{p}{:}\PY{p}{,} \PY{n}{l}\PY{p}{]} \PY{p}{)}
        
        \PY{c+c1}{\PYZsh{}Setting the value for the lower triangle}
        \PY{n}{X}\PY{p}{[}\PY{n}{l}\PY{p}{,} \PY{n}{k}\PY{p}{]} \PY{o}{=} \PY{n}{X}\PY{p}{[}\PY{n}{k}\PY{p}{,} \PY{n}{l}\PY{p}{]}
\end{Verbatim}
\end{tcolorbox}



    That covers the \(\boldsymbol{X}\) matrix.





    \hypertarget{constructing-the-boldsymboly-matrix}{%
\paragraph{\texorpdfstring{Constructing the \(\boldsymbol{Y}\)
Matrix}{Constructing the \textbackslash{}boldsymbol\{Y\} Matrix}}\label{constructing-the-boldsymboly-matrix}}





    Now let's construct the matrix:

\[
\boldsymbol{Y} =
\begin{pmatrix}
\langle{y}\rangle\\
\langle{x_1 y}\rangle\\
\langle{x_2 y}\rangle\\
\vdots\\
\langle{x_m y}\rangle\\
\end{pmatrix}
\]





    This is fairly straight forward, with

\[
\boldsymbol{Y}_{0, 0} = \langle{y}\rangle
\]

and

\[
\boldsymbol{Y}_{k, 0} = \langle{x_k y}\rangle ~~~~\text{where}~ k = 1, \dots, m
\]



    \begin{tcolorbox}[breakable, size=fbox, boxrule=1pt, pad at break*=1mm,colback=cellbackground, colframe=cellborder]
\prompt{In}{incolor}{50}{\boxspacing}
\begin{Verbatim}[commandchars=\\\{\}]
\PY{c+c1}{\PYZsh{}Creating the Y column matrix:}
\PY{n}{Y} \PY{o}{=} \PY{n}{np}\PY{o}{.}\PY{n}{matrix}\PY{p}{(} \PY{n}{np}\PY{o}{.}\PY{n}{zeros}\PY{p}{(} \PY{p}{(}\PY{n}{var\PYZus{}count}\PY{p}{,} \PY{l+m+mi}{1}\PY{p}{)} \PY{p}{)} \PY{p}{)}

\PY{c+c1}{\PYZsh{}First entry}
\PY{n}{Y}\PY{p}{[}\PY{l+m+mi}{0}\PY{p}{,} \PY{l+m+mi}{0}\PY{p}{]} \PY{o}{=} \PY{n}{np}\PY{o}{.}\PY{n}{mean}\PY{p}{(}\PY{n}{data}\PY{p}{[}\PY{p}{:}\PY{p}{,}\PY{l+m+mi}{0}\PY{p}{]}\PY{p}{)}

\PY{c+c1}{\PYZsh{}The remainder of the entries}
\PY{k}{for} \PY{n}{k} \PY{o+ow}{in} \PY{n+nb}{range}\PY{p}{(}\PY{l+m+mi}{1}\PY{p}{,} \PY{n}{var\PYZus{}count}\PY{p}{)}\PY{p}{:}
    \PY{n}{Y}\PY{p}{[}\PY{n}{k}\PY{p}{,} \PY{l+m+mi}{0}\PY{p}{]} \PY{o}{=} \PY{n}{np}\PY{o}{.}\PY{n}{mean}\PY{p}{(} \PY{n}{data}\PY{p}{[}\PY{p}{:}\PY{p}{,} \PY{n}{k}\PY{p}{]} \PY{o}{*} \PY{n}{data}\PY{p}{[}\PY{p}{:}\PY{p}{,} \PY{l+m+mi}{0}\PY{p}{]} \PY{p}{)}
\end{Verbatim}
\end{tcolorbox}



    \hypertarget{finding-matrix-boldsymbola-or-solving-for-the-a_j}{%
\paragraph{\texorpdfstring{Finding Matrix \(\boldsymbol{A}\) (Or Solving
For the
\(a_j\))}{Finding Matrix \textbackslash{}boldsymbol\{A\} (Or Solving For the a\_j)}}\label{finding-matrix-boldsymbola-or-solving-for-the-a_j}}





    Lastly, to solve for our \(a_j\) values, we consider the matrix:

\[
\boldsymbol{A}=
\begin{pmatrix}
a_0 \\
a_1\\
a_2\\
\vdots\\
a_m\\
\end{pmatrix}
\]





    This fits into the matrix equation

\[
\boldsymbol{X}\boldsymbol{A} = \boldsymbol{Y}
\]

where we've already constructed \(\boldsymbol{X}\) and
\(\boldsymbol{Y}\). All that's left is to solve the equation be
inverting \(\boldsymbol{X}\):

\[
\boldsymbol{A} = \boldsymbol{X}^{-1} \boldsymbol{Y}
\]

To achieve this numerically, we simply take the inverse of \texttt{X},
\texttt{X.I}:



    \begin{tcolorbox}[breakable, size=fbox, boxrule=1pt, pad at break*=1mm,colback=cellbackground, colframe=cellborder]
\prompt{In}{incolor}{51}{\boxspacing}
\begin{Verbatim}[commandchars=\\\{\}]
\PY{c+c1}{\PYZsh{}Finding A:}

\PY{n}{A} \PY{o}{=} \PY{n}{X}\PY{o}{.}\PY{n}{I}\PY{o}{*}\PY{n}{Y}

\PY{n+nb}{print}\PY{p}{(}\PY{n}{A}\PY{p}{)}
\end{Verbatim}
\end{tcolorbox}

    \begin{Verbatim}[commandchars=\\\{\}]
[[-2.14515885]
 [-3.11733284]
 [ 1.48566643]]
    \end{Verbatim}



    As you can see are results agree with the specific solution for the the
case of 3 variables above.





    \hypertarget{putting-it-all-together}{%
\paragraph{Putting it all together:}\label{putting-it-all-together}}

Let's gather all of the code cells together into a single script. We
will also merge the loops together for efficiency:



    \begin{tcolorbox}[breakable, size=fbox, boxrule=1pt, pad at break*=1mm,colback=cellbackground, colframe=cellborder]
\prompt{In}{incolor}{24}{\boxspacing}
\begin{Verbatim}[commandchars=\\\{\}]
\PY{k+kn}{import} \PY{n+nn}{numpy} \PY{k}{as} \PY{n+nn}{np}

\PY{c+c1}{\PYZsh{}Reading the data}
\PY{n}{data} \PY{o}{=} \PY{n}{np}\PY{o}{.}\PY{n}{loadtxt}\PY{p}{(}\PY{l+s+s1}{\PYZsq{}}\PY{l+s+s1}{data/cepheid\PYZus{}data.csv}\PY{l+s+s1}{\PYZsq{}}\PY{p}{,} \PY{n}{delimiter} \PY{o}{=} \PY{l+s+s1}{\PYZsq{}}\PY{l+s+s1}{,}\PY{l+s+s1}{\PYZsq{}}\PY{p}{,} \PY{n}{skiprows} \PY{o}{=} \PY{l+m+mi}{1}\PY{p}{)}

\PY{c+c1}{\PYZsh{} Swapping data[:, 0] and data[:, 1]}
\PY{c+c1}{\PYZsh{} Note that this is particular to the data file we are using}
\PY{c+c1}{\PYZsh{} np.copy is necessaary as arrays are not passed as values by default but as reference}
\PY{n}{data}\PY{p}{[}\PY{p}{:}\PY{p}{,} \PY{l+m+mi}{0}\PY{p}{]}\PY{p}{,} \PY{n}{data}\PY{p}{[}\PY{p}{:}\PY{p}{,} \PY{l+m+mi}{1}\PY{p}{]} \PY{o}{=} \PY{n}{np}\PY{o}{.}\PY{n}{copy}\PY{p}{(}\PY{n}{data}\PY{p}{[}\PY{p}{:}\PY{p}{,} \PY{l+m+mi}{1}\PY{p}{]}\PY{p}{)}\PY{p}{,} \PY{n}{np}\PY{o}{.}\PY{n}{copy}\PY{p}{(}\PY{n}{data}\PY{p}{[}\PY{p}{:}\PY{p}{,} \PY{l+m+mi}{0}\PY{p}{]}\PY{p}{)}

\PY{c+c1}{\PYZsh{}Creating empty X and Y matrices}
\PY{n}{var\PYZus{}count} \PY{o}{=} \PY{n}{data}\PY{o}{.}\PY{n}{shape}\PY{p}{[}\PY{l+m+mi}{1}\PY{p}{]}

\PY{n}{X} \PY{o}{=} \PY{n}{np}\PY{o}{.}\PY{n}{matrix}\PY{p}{(}\PY{n}{np}\PY{o}{.}\PY{n}{ones}\PY{p}{(} \PY{p}{(}\PY{n}{var\PYZus{}count}\PY{p}{,} \PY{n}{var\PYZus{}count}\PY{p}{)} \PY{p}{)}\PY{p}{)}
\PY{n}{Y} \PY{o}{=} \PY{n}{np}\PY{o}{.}\PY{n}{matrix}\PY{p}{(} \PY{n}{np}\PY{o}{.}\PY{n}{zeros}\PY{p}{(} \PY{p}{(}\PY{n}{var\PYZus{}count}\PY{p}{,} \PY{l+m+mi}{1}\PY{p}{)} \PY{p}{)} \PY{p}{)}

\PY{c+c1}{\PYZsh{}Filling the X and Y matrices}

\PY{n}{Y}\PY{p}{[}\PY{l+m+mi}{0}\PY{p}{,} \PY{l+m+mi}{0}\PY{p}{]} \PY{o}{=} \PY{n}{np}\PY{o}{.}\PY{n}{mean}\PY{p}{(}\PY{n}{data}\PY{p}{[}\PY{p}{:}\PY{p}{,}\PY{l+m+mi}{0}\PY{p}{]}\PY{p}{)}

\PY{k}{for} \PY{n}{k} \PY{o+ow}{in} \PY{n+nb}{range}\PY{p}{(}\PY{l+m+mi}{1}\PY{p}{,} \PY{n}{var\PYZus{}count}\PY{p}{)}\PY{p}{:}
    \PY{c+c1}{\PYZsh{}First row and column of X}
    \PY{n}{X}\PY{p}{[}\PY{l+m+mi}{0}\PY{p}{,} \PY{n}{k}\PY{p}{]} \PY{o}{=} \PY{n}{np}\PY{o}{.}\PY{n}{mean}\PY{p}{(}\PY{n}{data}\PY{p}{[}\PY{p}{:}\PY{p}{,} \PY{n}{k}\PY{p}{]}\PY{p}{)}
    \PY{n}{X}\PY{p}{[}\PY{n}{k}\PY{p}{,} \PY{l+m+mi}{0}\PY{p}{]} \PY{o}{=} \PY{n}{X}\PY{p}{[}\PY{l+m+mi}{0}\PY{p}{,} \PY{n}{k}\PY{p}{]}
    
    \PY{c+c1}{\PYZsh{}Y}
    \PY{n}{Y}\PY{p}{[}\PY{n}{k}\PY{p}{,} \PY{l+m+mi}{0}\PY{p}{]} \PY{o}{=} \PY{n}{np}\PY{o}{.}\PY{n}{mean}\PY{p}{(} \PY{n}{data}\PY{p}{[}\PY{p}{:}\PY{p}{,} \PY{n}{k}\PY{p}{]} \PY{o}{*} \PY{n}{data}\PY{p}{[}\PY{p}{:}\PY{p}{,} \PY{l+m+mi}{0}\PY{p}{]} \PY{p}{)}
    
    \PY{c+c1}{\PYZsh{}Inner matrix of X}
    \PY{k}{for} \PY{n}{l} \PY{o+ow}{in} \PY{n+nb}{range}\PY{p}{(}\PY{n}{k}\PY{p}{,} \PY{n}{var\PYZus{}count}\PY{p}{)}\PY{p}{:}
        \PY{n}{X}\PY{p}{[}\PY{n}{k}\PY{p}{,} \PY{n}{l}\PY{p}{]} \PY{o}{=} \PY{n}{np}\PY{o}{.}\PY{n}{mean}\PY{p}{(} \PY{n}{data}\PY{p}{[}\PY{p}{:}\PY{p}{,} \PY{n}{k}\PY{p}{]} \PY{o}{*} \PY{n}{data}\PY{p}{[}\PY{p}{:}\PY{p}{,} \PY{n}{l}\PY{p}{]} \PY{p}{)}
        \PY{n}{X}\PY{p}{[}\PY{n}{l}\PY{p}{,} \PY{n}{k}\PY{p}{]} \PY{o}{=} \PY{n}{X}\PY{p}{[}\PY{n}{k}\PY{p}{,} \PY{n}{l}\PY{p}{]}

\PY{c+c1}{\PYZsh{}Calculating A}

\PY{n}{A} \PY{o}{=} \PY{n}{X}\PY{o}{.}\PY{n}{I}\PY{o}{*}\PY{n}{Y}

\PY{n+nb}{print}\PY{p}{(}\PY{n}{A}\PY{p}{)}
\end{Verbatim}
\end{tcolorbox}

    \begin{Verbatim}[commandchars=\\\{\}]
[[-2.14515885]
 [-3.11733284]
 [ 1.48566643]]
    \end{Verbatim}


    % Add a bibliography block to the postdoc
    
    

    
\end{document}
